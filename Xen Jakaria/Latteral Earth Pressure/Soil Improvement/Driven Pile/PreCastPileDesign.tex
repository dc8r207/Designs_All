\documentclass[12pt]{article}
\usepackage{graphicx} %required for inserting images
\usepackage{array}
\usepackage{lipsum}
\usepackage{amsmath}
\title{GEOTECHNICAL DESIGN OF DRIVEN PILES UNDER AXIAL LOADS}
\author{Sakib Bin Rafi Tonmoy,Jakaria Pervez}
\date{03/04/2023}
\begin{document}
\maketitle
\section{Introduction}
3(Three) Methods shall be discused with example.
\begin{enumerate}
\item API:American Petroleum Institute Procedure
\item 2.FHWA:U.S. Federal Highway Administration Procedure
\item 3.U.S. Army Corps Method
\end{enumerate}
API gives lowest Capacity.It is conservative.So It shall be used in this note in detailed.
\section{API Procedure}
%\lipsum[2]
%writing necessary equation


\begin{math}
\E=mc^2
\end{math}


\begin{center}
\begin{table}
\begin{tabular}{|m{10cm}|m{2cm}|m{4cm}|}
\hline
soil & \delta,degrees & Limiting f,kips/fit^2(kPa) \\
\hline
Very loose & 15 & 1.0 (47.8)\\
\hline
Loose & 20 & 1.4 (67.0)\\
\hline
Medium & 25 & 1.7 (83.1)\\
\hline
Dense & 30 & 2.0 (95.5)\\
\hline
\end{tabular}
\caption{\label{f_non_cohesive}uideline for Side Friction in Siliceous Soil}
\end{table}

\end{center}
\end{document}